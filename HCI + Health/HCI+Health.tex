\documentclass{beamer}

\mode<presentation> {
\usetheme[secheader]{Madrid}
\usecolortheme{seahorse}
\useinnertheme{circles}
}
\usepackage{graphicx} % Allows including images
\usepackage{booktabs} % Allows the use of \toprule, \midrule and \bottomrule in tables
\usepackage{tikz}
\usepackage{url}

\hypersetup{colorlinks=true,linkcolor=blue,urlcolor=blue}
\urlstyle{rm}


%----------------------------------------------------------------------------------------
%	TITLE PAGE
%----------------------------------------------------------------------------------------

\title[HCI and Health]{HCI and Health} % The short title appears at the bottom of every slide, the full title is only on the title page

\author{Chaklam Silpasuwanchai} % Your name
\institute[AIT] % Your institution as it will appear on the bottom of every slide, may be shorthand to save space
{
Asian Institute of Technology \\ % Your institution for the title page
\medskip
\textit{chaklam@ait.asia} % Your email address
}
\date{} % Date, can be changed to a custom date

\begin{document}

\begin{frame}
\titlepage % Print the title page as the first slide
\end{frame}

\begin{frame}
\frametitle{Overview} % Table of contents slide, comment this block out to remove it
\tableofcontents % Throughout your presentation, if you choose to use \section{} and \subsection{} commands, these will automatically be printed on this slide as an overview of your presentation
\end{frame}

\AtBeginSection[]
{
\begin{frame}<beamer> 
\tableofcontents[currentsection]  % show TOC and highlight current section
\end{frame}
}

%----------------------------------------------------------------------------------------
%	PRESENTATION SLIDES
%----------------------------------------------------------------------------------------

%------------------------------------------------

\section{Introduction}

\begin{frame}
\frametitle{Introduction} 
\begin{itemize}
	\item \textbf{Health} is one important subdomain of HCI (see \href{https://chi2022.acm.org/for-authors/presenting/papers/selecting-a-subcommittee/} {subcommittees} in CHI)
	\item Covers \textbf{whole spectrum} of contributions involving health, wellness, including physical, mental, and emotional well-being, clinical environments, self-managements, and everyday wellness.
	\item The key emphasis is evaluated upon the impact to key \textbf{stakeholders}
	\item Let's look at some studies, mainly focusing on \textbf{empirical studies} (and less on only questionnaires/interview studies)...
\end{itemize}

\end{frame}


\section{Physiological Computing}

\begin{frame}
\frametitle{Blood Pressure Monitoring (Wang et al.,  CHI 2018)}


\footnote{Wang et al., \textit{Seismo: Blood Pressure Monitoring using Built-in Smartphone Accelerometer and Camera.}  Proc. CHI 2018}

\end{frame}

%----------------------------------------------------------------------------------------

\end{document} 
